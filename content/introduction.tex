\section{Introduction}

Multirotor UAVs have been a topic of great research interest over the course of past decade. The first mathematical model of quadrotor vehicle capable of vertical takeoff and landing has been presented in \cite{hamel2002quad}. Since then, researchers have been working on various control designs, payload transportation and attaching manipulators on such vehicles. This research allowed UAVs to execute various tasks and interact with the environment. However, the common denominator in all that research is treating the payload and movement of the manipulator as a disturbance. Such approach puts all the effort for canceling these effects on the controller. Although we are used to treating these effects as disturbances, a question arises: why not use them to our advantage? This underdog idea has not reached much attention in research community and it is one of the greatest motivators for this paper. The main goal is to develop a geometric controller capable of utilizing manipulator movement in order to track attitude. 

Geometric control concept is well known in aerial robotics community, and has previously been applied for classic quadrotor vehicles in \cite{LeeClanak4}, \cite{LeeClanak3}, \cite{LeeClanak1}, \cite{Kumar}. The UAV is considered to be in plus configuration (body-fixed frame is aligned with UAV arms) and the CoG coincides with the body of the UAV. Furthermore, the model of the UAV is augmented with the CoG dynamics so we can exploit manipulator movement for attitude tracking. In our previous work we designed and implemented the Moving Mass Control concept (MMC) \cite{movingMass1}. In that paper we use a standard quadrotor UAV with mounted moving masses on the arms of the UAV while the control structure uses standard PID blocks. This is a novel concept first developed in \cite{movingMass2} with attitude control considered in \cite{movingMass3}. Up to this point, nonlinear geometric control has not been applied to the moving-mass controlled UAV.