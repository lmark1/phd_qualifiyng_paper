\section{Introduction}

Multirotor UAVs have been a topic of great research interest over the course of past decade. The first mathematical model of quadrotor vehicle capable of vertical takeoff and landing has been presented in \cite{hamel2002quad}. Since then, researchers have been working on various control designs with applications in payload transportation, formation flight, exploration, environment interaction etc.

In traditional UAV control methods the configuration is considered in Euclidean space as $\zeta = [x, y, z, \phi, \theta, \psi] \in \mathbb{R}^6$, where the first three components describe the UAV position in the inertial frame, while the latter three describe its orientation in the inertial frame \cite{uavModel}.

On the other hand, in geometric control the UAV configuration is considered a rigid body described by the location of its center of mass in $\mathbb{R}^3$ and attitude represented by the $3\times3$ orthonormal matrix. To discover the underlying structure of such a configuration geometric mechanics principles need to be applied. Originally introduced by Jurdjevic 1996. \cite{jurdjevic_1996}, Bloch 2004.\cite{bloch}, Bullo and Lewis \cite{bulloBook}, Lee 2008. \cite{Lee2008ComputationalGM}, Lee et al. 2018. \cite{LeeModel}, geometric mechanics is a modern description of classical mechanics from the perspective of differential geometry. \\
Therefore, it can now be stated that a set of $3\times3$ orthonormal matrices representing UAV attitude is a manifold, as there exists a local bijective map towards an $\mathbb{R}^3$ Euclidean space, i.e. each $3\times3$ matrix corresponds to a vector in $\mathbb{R}^3$ representing rotations around inertial frame axes. Assuming smooth maps and equipping the manifold with a group operator of matrix multiplication a Special orthogonal Lie group SO(3) consisting of $3\times3$ orhtonormal matrices is subsequently defined. The final configuration is then a combination between translation in $\mathbb{R}^3$ and rotation in SO(3) resulting in a Special Euclidean Lie group SE(3) consisting of $4\times4$ transformation matrices containing both rotation and translation components.

\todo[inline]{Napisati prednosti ovog nacina}