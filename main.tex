%%%%%%%%%%%%%%%%%%%%%%%%%%%%%%%%%%%%%%%%%%%%%%%%%%%%%%%%%%%%%%%%%%%%%%%%%%%%%%%%
%2345678901234567890123456789012345678901234567890123456789012345678901234567890
%        1         2         3         4         5         6         7         8

\documentclass[letterpaper, 10 pt, conference]{ieeeconf}  % Comment this line out
                                                          % if you need a4paper
%\documentclass[a4paper, 10pt, conference]{ieeeconf}      % Use this line for a4
                                                          % paper

\IEEEoverridecommandlockouts                              % This command is only
                                                          % needed if you want to
                                                          % use the \thanks command
\overrideIEEEmargins
% See the \addtolength command later in the file to balance the column lengths
% on the last page of the document

\pdfminorversion=4

% The following packages can be found on http:\\www.ctan.org
%\usepackage{graphics} % for pdf, bitmapped graphics files
%\usepackage{epsfig} % for postscript graphics files
%\usepackage{mathptmx} % assumes new font selection scheme installed
%\usepackage{times} % assumes new font selection scheme installed
%\usepackage{amsmath} % assumes amsmath package installed
%\usepackage{amssymb}  % assumes amsmath package installed
\usepackage{amsmath}    			% ams packages for mathematics environment    
\usepackage{amssymb}
\usepackage{amsfonts}
%\usepackage{amsthm}
\usepackage{balance}
\usepackage{graphicx}  				% Versatile graphics manipulation options

\usepackage[croatian]{babel}  % Croatian typographical rules and hyphenation patterns 
\usepackage[utf8]{inputenc}  	% Encoding of Croatian characters
\usepackage[T1]{fontenc}
\usepackage{ae,aecompl}     	% Type 1 fonts, similar to Computer Modern

\usepackage{microtype}				% Improves spacing

\usepackage{hyperref}
%\usepackage{subfig}
\usepackage{subcaption}
\captionsetup[figure]{labelsep=period}
\captionsetup[subfigure]{labelformat=simple} % default is 'parens'
\renewcommand\thesubfigure{Fig. \thefigure.\alph{subfigure})}
\usepackage{tabularx}
\usepackage{booktabs}
%\newcolumntype{C}{>{\centering\arraybackslash}X} % centered version of "X" type
\setlength{\extrarowheight}{1pt}
\usepackage{enumerate}				% Additional options for listing of items in enumerate environment
\usepackage{algorithm2e}			% Writing pseudo-code
\usepackage{todonotes}				% Adding todo items
\usepackage{dirtree}					% Simple display of directory tree
\usepackage{hyperref}					% Managing cross-referencing


\usepackage{hhline}
\usepackage{caption}
\usepackage{lmodern}
\usepackage{nccmath}
%\usepackage[keeplastbox]{flushend}
\usepackage{scalerel,stackengine}
\stackMath
\newcommand\reallywidehat[1]{%
	\savestack{\tmpbox}{\stretchto{%
			\scaleto{%
				\scalerel*[\widthof{\ensuremath{#1}}]{\kern.1pt\mathchar"0362\kern.1pt}%
				{\rule{0ex}{\textheight}}%WIDTH-LIMITED CIRCUMFLEX
			}{\textheight}% 
		}{2.4ex}}%
	\stackon[-6.9pt]{#1}{\tmpbox}%
}
\parskip 1ex

% highlighting
\usepackage{soul}
\usepackage{color}

\usepackage{graphicx}
\usepackage{transparent}

\definecolor{reviewer1}{rgb}{0.235, 0.471, 0.847}
\definecolor{reviewer2}{rgb}{0.416, 0.659, 0.31}
\definecolor{reviewer3}{rgb}{0.902, 0.568, 0.22}
\definecolor{redc}{rgb}{1.0, 0, 0}
\DeclareRobustCommand{\hlrevone}[1]{{\textcolor{reviewer1}{#1}}}
\DeclareRobustCommand{\hlrevtwo}[1]{{\textcolor{reviewer2}{#1}}}
\DeclareRobustCommand{\hlrevthree}[1]{{\textcolor{reviewer3}{#1}}}
\DeclareRobustCommand{\hlremove}[1]{{\textcolor{redc}{#1}}}

%calligraphy packages
\usepackage{calrsfs}
\DeclareMathAlphabet{\pazocal}{OMS}{zplm}{m}{n}
\newcommand{\Ca}{\pazocal{C}}
\newcommand{\Oa}{\pazocal{O}}
\newcommand{\Va}{\pazocal{V}}
\newcommand{\Ua}{\pazocal{U}}
\newcommand{\Aa}{\pazocal{A}}
\newcommand{\Ta}{\pazocal{T}}
\newcommand{\La}{\pazocal{L}}
\newcommand{\Ja}{\pazocal{J}}

\graphicspath{{./figures/}}
\usepackage{float}
\usepackage{multicol}
\title{\LARGE \bf
State-of-the-Art Survey of Manifold Based Control Methods for Unmanned Aerial Vehicles
}
\author{Lovro Markovic \\
	University of Zagreb, Faculty of Electrical Engineering and Computing \\
	Laboratory for Robotics and Intelligent Control Systems (LARICS) \\
	Unska 3, 10000 Zagreb \\
	Email: lovro.markovic@fer.hr
}

\makeatletter
\newcommand{\removelatexerror}{\let\@latex@error\@gobble}
\newcommand{\mb}[1]{\boldsymbol{#1}}
\newcommand{\norm}[1]{\left\lVert#1\right\rVert}

\makeatother

\begin{document}
\maketitle

\thispagestyle{empty}
\pagestyle{empty}


%%%%%%%%%%%%%%%%%%%%%%%%%%%%%%%%%%%%%%%%%%%%%%%%%%%%%%%%%%%%%%%%%%%%%%%%%%%%%%%%
\begin{abstract}
This paper presents a survey of manifold based control methods for unmanned aerial vehicles (UAVs). 
Computational geometric mechanics and passive decomposition are the two general frameworks, with an embedded manifold structure, used as a starting point in this paper. Although notions of these frameworks can be applied on any rigid body, this survey is limited to exploring research mainly relating to UAVs. These frameworks are used to tackle a wide variety of problems. The former solves issues of trajectory tracking, robust control, cooperative-load transportation etc. while the latter mainly concerns with environment interaction, exploration and teleoperation. State-of-the-art survey of the mentioned frameworks and their applications is presented in the remainder of this paper.
\end{abstract}

\section{Introduction}

Multirotor UAVs have been a topic of great research interest over the course of past decade. The first mathematical model of quadrotor vehicle capable of vertical takeoff and landing has been presented in \cite{hamel2002quad}. Since then, researchers have been working on various control designs with applications in payload transportation, formation flight, exploration, environment interaction etc.\\
In traditional UAV control methods the configuration is considered in Euclidean space as $\zeta = [x, y, z, \phi, \theta, \psi] \in \mathbb{R}^6$, where the first three components describe the UAV position in the inertial frame, while the latter three describe its orientation in the inertial frame \cite{uavModel}.

\noindent \textbf{Geometric Control}, on the other hand, considers the UAV configuration as a rigid body described by the location of its center of mass in $\mathbb{R}^3$ and attitude represented by the $3\times3$ orthonormal matrix. To discover the underlying structure of such a configuration geometric mechanics principles need to be applied. Originally introduced by Jurdjevic 1996. \cite{jurdjevic_1996}, Bloch 2004.\cite{bloch}, Bullo and Lewis \cite{bulloBook}, T. Lee 2008. \cite{Lee2008ComputationalGM}, T. Lee et al. 2018. \cite{LeeModel}, geometric mechanics is a modern description of classical mechanics from the perspective of differential geometry. \\
Therefore, it can now be stated that a set of $3\times3$ orthonormal matrices representing UAV attitude is a manifold, as there exists a local bijective map towards an $\mathbb{R}^3$ Euclidean space, i.e. each $3\times3$ matrix corresponds to a vector in $\mathbb{R}^3$ representing rotations around inertial frame axes. Assuming smooth maps and equipping the manifold with a group operator of matrix multiplication a Special orthogonal Lie group SO(3) consisting of $3\times3$ orhtonormal matrices is subsequently defined. The final configuration is then a combination between translation in $\mathbb{R}^3$ and rotation in SO(3) resulting in a Special Euclidean Lie group SE(3) consisting of $4\times4$ transformation matrices containing both rotation and translation components. \\ 
Furthermore the UAV dynamics are, within this approach, considered as a Lagrangian or Hamiltonian system along with its unique properties: energy is conserved in absence of non-conservative forces; momentum map, a generalized notion of linear and angular momentum, with the system symmetry is preserved.

The main advantage of geometric control is the coordinate-free approach. Representing objects wrt. coordinate frames is often confusing and error prone. This approach only relies on compact expressions wrt. the inertial coordinate frame. To further elaborate, let the UAV attitude be taken into consideration. Expressed globally in the SO(3) configuration manifold no issues arise. If the parametrization to one of 24 types of Euler angles is applied, singularities may occur when multiple Euler angle sets correspond to a single SO(3) configuration. When expressing the attitude in quaternions sign ambiguities occur as there are always two possible quaternion solutions representing the SO(3) configuration \cite{euler}.

\noindent \textbf{Passive Decomposition} is the second control framework considered in this survey. Previous research... \cite{LeePassive} \cite{passive6} \cite{passive8} \cite{passive12}

%%%%%%%%%%%%%%%%%%%%%%%%%%%%%%%%%%%%%%%%%%%%%%%%%%%%%%%%%%%%%%%%%%%%%%%%%%%%%%%%
%\section*{APPENDIX} \label{sec:appendix}
%\input{content/appendix.tex}

%Appendixes should appear before the acknowledgment.

\section*{ACKNOWLEDGMENT}

This research was supported by European Commission Horizon 2020 Programme through project under G. A. number 810321, named Twinning coordination action for spreading excellence in Aerial Robotics - AeRoTwin \cite{AEROTWINweb} and through project under G. A. number 820434, named ENergy aware BIM Cloud Platform in a COst-effective Building REnovation Context - ENCORE \cite{ENCOREweb}.
%%%%%%%%%%%%%%%%%%%%%%%%%%%%%%%%%%%%%%%%%%%%%%%%%%%%%%%%%%%%%%%%%%%%%%%%%%%%%%%%

%\nocite{*}
\bibliographystyle{ieeetr}
\bibliography{bibliography/Mendeley}
\end{document}
